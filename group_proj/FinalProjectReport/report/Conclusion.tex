%This is the only file to have 2 different sections in it
%1 section is for the Conclusion
%The other is for the Future work
%
%They were combined because they are deeply connected

\section{Conclusion}
We found that the best performing algorithm was CNN with a 76\% accuracy and the least amount of under/overfitting. Although evaluation of the CNN loss functions suggests that the learning process is slightly unstable, there were still signs of convergence and the overall metrics indicated good performance. Throughout this project we were able to learn about the different approaches to image classification and can understand why CNN is a popular choice for the task. The architecture of CNN models allow for efficient handling of image classification aspects as the layers within the architecture capture and learn patterns in images. Since image classification involves the recognition of abstract and hierarchical features, CNN out performs SVM as SVM tends to perform better alongside intricate feature engineering and lacks the ability to learn patterns over a spatial hierarchy. SVM is also known to struggle with large datasets, performing inefficiently and experiencing futile and long training times. It is interesting to find that CNN seemed to have the opposite problem, finding our dataset to be on the smaller side and performing better when more variance through data augmentation was introduced. The performance of our pre-trained VGG16 model also indicated potential performance improvement on a larger dataset where a lengthy training time would be more efficient than that of SVM.

\section{Future Work}

Some improvements that can be done on our project would be to:
\begin{itemize}
    \item Increase dataset size and diversity of images
    \begin{itemize}
        \item A larger dataset may allow CNN models to achieve learning stability, increasing overall performance
        \item This would particularly be useful while applying the VGG16 model, as it is more computationally expansive and was pretrained on larger datasets
    \end{itemize}
    \item For CNN models built from scratch, more layers, filters, and neurons may need to be added to increase complexity, ensuring that the intricate features and patterns of the images were captured
    \item Stricter set of criteria for clean images
    \item Additional augmentations of images
    \item For SVM, use a smaller subset of the data to perform training faster
\end{itemize}