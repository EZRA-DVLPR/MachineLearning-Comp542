\documentclass[a4paper,10pt]{article}
\usepackage{graphicx}
\usepackage[english]{babel}
\usepackage[latin1]{inputenc}
\usepackage{cite}
\usepackage{hyperref}
\usepackage[table]{xcolor}
\usepackage{geometry}
\usepackage{subcaption}
\usepackage{newclude}
\usepackage{indentfirst}

\geometry{margin=1in} % Adjust margins as needed
%
\begin{document}
%
  \title{Animal Image Classification}
  \author{
  Isaiah Martinez \\ CSUN \\ Computer Science Department \\ isaiah.martinez.891@my.csun.edu
  \and
  Joycelyn Tuazon \\ CSUN \\ Computer Science Department \\ joycelyn.tuazon.251@my.csun.edu}
          
  \date{12/02/2023}

  \maketitle
   
  \tableofcontents
 
  % to import a new page into this document, use
  % `\include{FILENAME}`
  %     Note: FILENAME does NOT include the `.tex` file extension
  % compilation follows as normal

  \newpage

  \include*{Intro}
  \include*{RelatedWorks}
  \include*{Methods}
  \include*{Evaluation}
  \include*{Conclusion}

    % bibliography / References

  \newpage

  \begin{thebibliography}{}

    \bibitem{AIAdv} \href{https://ai100.stanford.edu/gathering-strength-gathering-storms-one-hundred-year-study-artificial-intelligence-ai100-2021-1/sq2#:~:text=In%20the%20last%20five%20years,and%20integration%20of%20vision%20and}{Stanford: What are the most important Advances in AI?}

    \bibitem{publicOpinion} \href{https://www.pewresearch.org/short-reads/2023/08/28/growing-public-concern-about-the-role-of-artificial-intelligence-in-daily-life/}{PRC: Growing public concern about the role of artificial intelligence in daily life}
  
    \bibitem{Dataset} \href{https://www.kaggle.com/datasets/ashishsaxena2209/animal-image-datasetdog-cat-and-panda}{Kaggle: Animal Image Dataset(DOG, CAT and PANDA)}

    \bibitem{SVM} \emph{Comparison of Support Vector Machine Classifier and Naïve Bayes Classifier on Road Surface Type Classification} by Marianingsih and Utaminingrum.

    \bibitem{CNNPerformance} \emph{How Does the Data set Affect CNN-based Image Classification Performance?} by Luo, Li, Wang, et. al.

    \bibitem{AnimalSpecies1} \emph{An Enhanced Animal Species Classification and Prediction Engine using CNN} by Priya, Kalyan, et. al.
  
    \bibitem{AnimalSpecies2} \emph{Animal Species Image Classification} by Prudhivi, Krishna et. al.

    \bibitem{AnimalBreed} \emph{Animal Breed Classification and Prediction Using Convolutional Neural Network Primates as a Case Study} by Kamepalli, Kolli, and Bandaru

    \bibitem{Mosquito} \emph{CNN Architectures Performance Evaluation for Image Classification of Mosquito in Indonesia} by Amiruddin and Kadir
    
    \bibitem{VGG16Greyscale} \emph{An Approach to Run Pre-Trained Deep Learning Models on Grayscale Images} by Ahmad and Shin

    \bibitem{CLIP} \href{https://github.com/openai/CLIP/tree/main}{CLIP algorithm Github Page}

    \bibitem{SVMExplained} \href{https://www.ibm.com/docs/en/spss-modeler/saas?topic=models-how-svm-works}{SVM explained from IBM.}
    
    \bibitem{CNNGuide} \emph{A Review of Convolutional Neural Networks, its Variants and Applications} by Shruti and Rekha

    \bibitem{Gradient} \href{https://machinelearningmastery.com/gradient-in-machine-learning/}{Gradient Explained}

    \bibitem{RN50} \href{https://viso.ai/deep-learning/resnet-residual-neural-network/}{Res Net 50 Explained}
    
    \bibitem{VGG16Explained} \href{https://www.kaggle.com/code/blurredmachine/vggnet-16-architecture-a-complete-guide}{VGG16 Explained}

    \bibitem{Gradient} \href{https://machinelearningmastery.com/gradient-in-machine-learning/}{Gradient Explained}





    

  \end{thebibliography}

\end{document}